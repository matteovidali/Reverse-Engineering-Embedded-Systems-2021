\documentclass[12pt,twoside]{article}

\newif\ifsolutions
\solutionstrue

\usepackage[colorlinks, urlcolor=blue]{hyperref}
%\usepackage{breakurl}
\usepackage{amsmath}
\usepackage{amsfonts}
\usepackage{graphicx}
\usepackage{fancyvrb}
% If using minted, must call pdflatex with -shell-escape
% \usepackage{minted}
\usepackage{parskip}
\newcommand{\profs}{Prepared by: Austin Roach}
\newcommand{\shortsubj}{Reverse-engineering embedded systems}
\newcommand{\longsubj}{ENGR-E399/599}
\newcommand{\term}{Spring 2022}
\newcommand{\organization}{Indiana University}
\newcommand{\assignmentname}{Syllabus}
\newcommand{\assignmentdate}{January 13, 2022}

\newlength{\toppush}
\setlength{\toppush}{2\headheight}
\addtolength{\toppush}{\headsep}

\newcommand{\htitle}{\noindent\vspace*{-\toppush}\newline\parbox{6.5in}
{\textit{\longsubj}\hfill\newline
\organization\hfill\term\newline
\profs\hfill \assignmentdate\vspace*{-.5ex}\newline
\mbox{}\hrulefill\mbox{}}\vspace*{1ex}\mbox{}\newline
\begin{center}
{\Large\bf \assignmentname}\\
\end{center}}

\newcommand{\handout}{\thispagestyle{empty}
\markboth{\shortsubj: \assignmentname}{\shortsubj: \assignmentname}
\pagestyle{myheadings}}

\setlength{\oddsidemargin}{0pt}
\setlength{\evensidemargin}{0pt}
\setlength{\textwidth}{6.5in}
\setlength{\topmargin}{0in}
\setlength{\textheight}{8.5in}

% Allow breaks at underscores
\renewcommand\_{\textunderscore\allowbreak}

% Superscript footnote comma separator
\newcommand{\fnsep}{\textsuperscript{,}}

\begin{document}
\htitle
\handout
\setlength{\parindent}{0pt}

This course provides an introduction to embedded systems reverse engineering.
Its focus is the practical exploration of the process of reverse engineering
using tools and techniques relevant to embedded systems. This course will give
experience with:
\begin{itemize}
\item Analyzing embedded systems architectures
\item Identifying and interacting with debug interfaces and external
communication interfaces
\item Extracting information for analysis from nonvolatile memories and
firmware update files
\item Disassembling, decompiling, and analyzing executable code for various
microcontrollers and microprocessors
\item Emulating firmware
\item Identifying appropriate points for analysis given a reverse engineering
goal
\end{itemize}
The course will explore a range of embedded systems architectures, from those
based on 8-bit microcontrollers to those based on microprocessors running
embedded multitasking operating systems.

\medskip

\hrulefill

\medskip

\section{Instructor}

Austin Roach\\
ahroach@iu.edu

\section{Schedule}

{\bf Lecture:}
Thursday 4:55PM-7:25PM, 4012 Luddy Hall\\
Lectures will be recorded to allow students to view lectures for which they are
absent and to review lecture content as needed.

{\bf Office hours}:
Tuesday 7:30PM-8:30PM, 2032 Luddy Hall\\

If students are unable to attend office hours, appointments with the instructor
on Zoom can be scheduled upon request.

A tentative schedule of lecture topics is:

\begin{center}
\renewcommand{\arraystretch}{1.5}
\begin{tabular}{|l|l|}
\hline
{\bf Date} & {\bf Topic}\\
\hline
{\bf 01/13/22} & Course introduction and overview; introduction to Ghidra\\
\hline
{\bf Online} & ARM ISA; ARM disassembly and code analysis\\
\hline
{\bf 01/20/22} & Identifying standard library functions: manual and automated techniques\\
\hline
{\bf 01/27/22} & ELFs; RE hints from linking and loading\\
\hline
{\bf 02/03/22} & Embedded Linux systems hardware/software architecture; \\
& analyzing filesystem images; reverse engineering using operating system abstractions\\
\hline
{\bf 02/10/22} & Dynamic analysis of embedded Linux systems; emulation\\
\hline
{\bf 02/17/22} & AVR architecture and ISA; AVR disassembly\\
\hline
{\bf 02/24/22} & AVR I/O interfaces; automating interaction with I/O interfaces\\
\hline
{\bf 03/03/22} & Microcontroller emulation\\
\hline
{\bf 03/10/22} & Serial protocols and interfaces: discovery and interaction\\
\hline
{\bf 03/24/22} & JTAG; interacting with debug access ports\\
\hline
{\bf 03/31/22} & STM32F4 architecture; analyzing object-oriented code\\
\hline
{\bf 04/07/22} & Firmware update files\\
\hline
{\bf 04/14/22} & Advanced analysis and RE approaches\\
\hline
{\bf 04/21/22} & Security threats and evaluation techniques\\
\hline
{\bf 04/28/22} & ENGR-E 599 student presentations: embedded systems RE case studies\\
\hline
\end{tabular}
\end{center}

\section{Prerequisites}

While there are no official prerequisites, students will benefit from some
familiarity with:
\begin{itemize}
\item electronics
\item programming in C and Python
\item assembly language in at least one architecture
\item microcontroller or bare-metal microprocessor programming
\item operating systems
\end{itemize}
Background in these areas will be provided as needed for the assignments. If a
student needs more background information in a particular topic area than is
provided during lecture,  they should inform the instructor so that additional
resources or instruction can be provided.

\section{Textbooks}

This course has no official textbook. Some generally useful references are
listed below. References relevant to particular topic areas will be presented
during class.

{\bf Practical Reverse Engineering: x86, x64, ARM, Windows Kernel, Reversing
Tools, and Obfuscation}\\
Bruce Dang, Alexandre Gazet, Elias Bachaalany, and S\'{e}bastian Josse\\
John Wiley \& Sons\\
ISBN-13: 978-1-118-78731-1\\
\url{https://iu.skillport.com/skillportfe/main.action?assetid=62680}

{\bf Reverse Engineering for Beginners}\\
Dennis Yurichev\\
Self-published\\
\url{https://beginners.re/}

{\bf Hacking the Xbox: An Introduction to Reverse Engineering}\\
Andrew ``bunnie'' Huang\\
No Starch Press\\
ISBN-13: 978-1-59327-029-2\\
\url{https://bunniefoo.com/nostarch/HackingTheXbox_Free.pdf}

{\bf The Ghidra Book}\\
Chris Eagle\\
No Starch Press\\
ISBN-13: 978-1-71850-102-7

{\bf The IDA Pro Book}\\
Chris Eagle\\
No Starch Press\\
ISBN-13: 978-1-59327-289-0\\
\url{https://iu.skillport.com/skillportfe/main.action?assetid=43616}

{\bf Practical IoT Hacking}\\
Fotios Chantzis, Ioannis Stais, Paulino Calderon, Evangelos Deirmentzoglou, and Beau Woods\\
No Starch Press\\
ISBN-13: 978-1-71850-090-7\\

\section{Lectures}

Lectures will be recorded for later viewing. Lecture materials, including
slides and any other resources used during the lecture, will be shared in a
course git repository.

\section{Assignments}

Assignment submissions should include a written description of the techniques
that the student used to complete the assignment and the details that were
learned through the analysis.  Assignment submissions may either be flat text
files or PDFs. Some assignments may also require the submission of supporting
artifacts, such as analysis files from the disassembler or source code for programs
that were written to interact with a system.

Late portions of an assignment will be accepted, but the points awarded for any
late portions will be reduced by 50\%. Even if an assignment is incomplete,
students are strongly encouraged to submit whatever they have completed by the
due date in order to maximize the awarded points.

\section{Policy on working together}

Cooperation on the assignments is accepted and encouraged. Any collaborators
must be acknowledged on assignment submissions, and assignment reports and
source code must be written independently. {\bf Duplication of another
student’s work is not permitted, and will result in a score of zero points for
the assignment.}

\section{Hardware}

Students will be provided with hardware to support some of the assignments.
Students should plan to return the hardware kits by the date of the final
class.

\section{Class participation}

Class participation accounts for a small fraction of the final score. Students
can show their continued engagement with the course material in a variety of
ways, such as by attending lectures, participating in discussions, attending
office hours, or corresponding with the instructor about assignments or course
topics.

For students in the ENGR-E 599 version of the course, full participation will
also include the delivery of a 10-15 minute presentation on an embedded systems
reverse engineering topic of the student's choice during the final class
period. This could be a summary of a research paper, details of a project found
online, or a description of a project undertaken by the student outside the
course. More information about the presentations will be provided mid-way
through the semester. If students would like help choosing a topic for
presentation, they are encouraged to contact the instructor.

\section{Grading}

Grading will be calculated from a combination of assignment grades and class
participation. There are no exams.

\begin{center}
\renewcommand{\arraystretch}{1.5}
\begin{tabular}{|c|c|}
\hline
{\bf Assignments} & {\bf Participation} \\
\hline
95\% & 5\% \\
\hline
\end{tabular}
\end{center}

In the event that alterations to the course plan require changes to this
weighting, any changes will be communicated on Canvas.

The cutoffs for assigning letter grades are as follows:

\begin{center}
\begin{tabular}{l|l}
\hline
{\bf Percentage} & {\bf Letter grade} \\
\hline
93\% & A \\
90\% & A- \\
87\% & B+ \\
83\% & B \\
80\% & B- \\
77\% & C+ \\
73\% & C \\
70\% & C- \\
67\% & D+ \\
63\% & D \\
60\% & D- \\
$<$60\% & F \\
\hline
\end{tabular}
\end{center}
The instructor reserves the right to decrease the grade cutoffs during the
course. In other words, the scale above represents the \emph{minimum} letter
grade that will be assigned, and the instructor may choose to restructure the
scale to assign better grades at their discretion. Any changes to the grading
scale will be communicated with a post to the course Canvas site.

Grades of A+ will be assigned for extraordinary achievement during the course.

\section{Communications}

Official announcements, course materials, and descriptions of labs and
assignments will be posted to Canvas. Students will also submit their
assignments on Canvas.

If a collaborative discussion forum (a Slack workspace, for example) would be
of interest to the class, one may be established with details for joining
posted to Canvas.

Please e-mail the instructor for any discussions related to personal matters,
grades, or course administration.

\section{Disability services for students}

If a student needs an accommodation for a disability, they should inform the
instructor during the first three weeks of the semester. Some aspects of this
course may be modified to facilitate the student's participation and progress.
Once the instructor is aware of a student's needs, they can work with the
Office of Disability Services for Students (DSS) to help determine appropriate
academic accommodations. Any information that a student provides is private and
confidential.

\end{document}

